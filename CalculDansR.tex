\documentclass[12pt]{article}
\usepackage{stmaryrd}
\usepackage{graphicx} % Pour l'insertion d'images
\usepackage{float}    % Pour contrôler précisément le placement
\usepackage[utf8]{inputenc}

\usepackage[french]{babel}
\usepackage[T1]{fontenc}
\usepackage{hyperref}
\usepackage{verbatim}

\usepackage{color, soul}

\usepackage{pgfplots}
\pgfplotsset{compat=1.15}
\usepackage{mathrsfs}

\usepackage{amsmath}
\usepackage{amsfonts}
\usepackage{amssymb}
\usepackage{tkz-tab}
\author{Destiné aux élèves de $2^{nd}$ L\\Lycée de Dindéfelo\\Présenté par M. BA}
\title{\textbf{Calcul dans $\mathbb{R}$}}
\date{\today}
\usepackage{tikz}
\usetikzlibrary{arrows, shapes.geometric, fit}

% Commande pour la couleur d'accentuation
\newcommand{\myul}[2][black]{\setulcolor{#1}\ul{#2}\setulcolor{black}}
\newcommand\tab[1][1cm]{\hspace*{#1}}

\usepackage[margin=2cm]{geometry}
\usepackage{eso-pic}         % Pour ajouter des éléments en arrière-plan

\usepackage{enumitem}
%---------------------------------------
% Définir un compteur pour les exemples
\newcounter{exemple}

% Définir la commande \exemple pour afficher un exemple numéroté
\newcommand{\exemple}{%
  \refstepcounter{exemple}% Incrémenter le compteur
  \textbf{\textcolor{orange}{Exemple \theexemple : }} \ignorespaces
}
%---------------------------------------
\newcounter{solution}

% Définir la commande \solutione pour afficher un solution numéroté
\newcommand{\solution}{%
  \refstepcounter{solution}% Incrémenter le compteur
  \textbf{\textcolor{orange}{Solution \thesolution : }} \ignorespaces
}
%---------------------------------------
\definecolor{myorange}{rgb}{1.0, 0.8, 0.0}

% Définir un compteur pour les exercices d'application
\newcounter{exerciceapp}

% Définir la commande pour afficher un exercice d'application numéroté
\newcommand{\exerciceapp}{%
  \refstepcounter{exerciceapp}%
  \textbf{\textcolor{myorange}{Exercice d'application \theexerciceapp :}} \ignorespaces
}
%--------------------------------------
% Définir un compteur pour les exercices d'application
\newcounter{correction}

% Définir la commande pour afficher un correction exercice d'application numéroté
\newcommand{\correction}{%
  \refstepcounter{correction}%
  \textbf{\textcolor{myorange}{Correction \thecorrection :}} \ignorespaces
}
%--------------------------------------
% Définir un compteur pour les remarque d'application
\newcounter{remarque}

%----------------------------------------
\definecolor{myorange1}{rgb}{1.0, 1.5, 0}
% Définir la commande pour afficher un remarque numéroté
\newcommand{\remarque}{%
  \refstepcounter{remarque}%
  \textbf{\textcolor{myorange1}{Remarque \theremarque :}} \ignorespaces
}
% Commande pour ajouter du texte en arrière-plan
\AddToShipoutPicture{
    \AtTextCenter{%
        \makebox[0pt]{\rotatebox{45}{\textcolor[gray]{0.9}{\fontsize{5cm}{5cm}\selectfont Pathé Gobel BA}}}
    }
}

\begin{document}
\maketitle
\section*{\underline{\textbf{\textcolor{red}{I. CALCULS SUR LES QUOTIENTS :}}}}
\subsection*{\underline{\textbf{\textcolor{red}{1.Règles de calcul :}}}}
\subsubsection*{\underline{\textbf{\textcolor{red}{a.Quotient de deux réels :}}}}
\textbf{\remarque}

\subsubsection*{\underline{\textbf{\textcolor{red}{b) Propriétés :}}}}

Soit $a$, $c$, $e$ des réels quelconques et $b$, $d$, $f$ des réels non nuls ($b \neq 0$, $d \neq 0$, $f \neq 0$) :

$\ast \ a = \frac{a}{b} \cdot b = 1 ; \frac{0}{b} = 0 ; \frac{a}{a} = 1 ; \frac{1}{\frac{a}{b}} = \frac{b}{a} \quad (\text{pour } a \neq 0).$

$-\frac{a}{b} = -\frac{a}{b} = \frac{-a}{b} = -\frac{a}{-b} = \frac{a}{-b}.$

\textbf{Addition-Soustraction}

$\frac{a}{b} + \frac{c}{b} =$

$\frac{a}{b} + c =$

$ \frac{a}{b} - \frac{c}{d} = \frac{a \times d - c \times b}{b \times d} $

$-\frac{a}{b} + c =$

$ \frac{a}{b} + \frac{c}{d} + \frac{e}{f} = \frac{a \times d \times f + c \times b \times f + e \times b \times d}{b \times d \times f}.$

\textbf{Produit}

$\frac{a}{b} \times \frac{c}{d}=$

$a \times \frac{c}{d}=$

\textbf{Quotient}

$\frac{\frac{a}{b}}{\frac{c}{d}}=$

$\frac{1}{\frac{c}{d}}=$

Egalité de deux quotients :

$\frac{a}{b}=\frac{c}{d} \implies$

\textbf{\exerciceapp}

Soit $a = 6$, $b = 3$, $c = 8$, $d = 4$, $e = 10$, et $f = 5$.

Utilise les propriétés des fractions pour effectuer les calculs suivants :

\begin{enumerate}
    \item \textbf{Addition-Soustraction :}
        \begin{enumerate}
            \item Calcule $\frac{a}{b} + \frac{c}{b}$.
            \item Calcule $\frac{a}{b} - \frac{c}{d}$.
            \item Calcule $\frac{a}{b} + \frac{c}{d} + \frac{e}{f}$.
        \end{enumerate}
    
    \item \textbf{Produit :}
        \begin{enumerate}
            \item Calcule $\frac{a}{b} \times \frac{c}{d}$.
            \item Calcule $a \times \frac{c}{d}$.
        \end{enumerate}
    
    \item \textbf{Quotient :}
        \begin{enumerate}
            \item Calcule $\frac{\frac{a}{b}}{\frac{c}{d}}$.
            \item Calcule $\frac{1}{\frac{c}{d}}$.
        \end{enumerate}
    
    \item \textbf{Égalité de deux quotients :}
    
    Vérifie si les quotients suivants sont égaux : $\frac{a}{b}$ et $\frac{c}{d}$. Justifie ta réponse.
\end{enumerate}

\textbf{\correction}

\begin{enumerate}
    \item \textbf{Addition-Soustraction :}
        \begin{enumerate}
            \item $\frac{6}{3} + \frac{8}{3} = \frac{6 + 8}{3} = \frac{14}{3}$.
            \item $\frac{6}{3} - \frac{8}{4} = \frac{6 \times 4 - 8 \times 3}{3 \times 4} = \frac{24 - 24}{12} = 0$.
            \item $\frac{6}{3} + \frac{8}{4} + \frac{10}{5} = 2 + 2 + 2 = 6$.
        \end{enumerate}
    
    \item \textbf{Produit :}
        \begin{enumerate}
            \item $\frac{6}{3} \times \frac{8}{4} = \frac{6 \times 8}{3 \times 4} = \frac{48}{12} = 4$.
            \item $6 \times \frac{8}{4} = 6 \times 2 = 12$.
        \end{enumerate}
    
    \item \textbf{Quotient :}
        \begin{enumerate}
            \item $\frac{\frac{6}{3}}{\frac{8}{4}} = \frac{2}{2} = 1$.
            \item $\frac{1}{\frac{8}{4}} = \frac{1}{2}$.
        \end{enumerate}
    
    \item \textbf{Égalité de deux quotients :}
    
    $\frac{6}{3} = 2$ et $\frac{8}{4} = 2$, donc $\frac{6}{3} = \frac{8}{4}$.
\end{enumerate}
%2) Identités remarquables :
%a) Rappels :
%b) Autres identités remarquables :
%\textbf{\exerciceapp}

%\textbf{\correction}
%II) PUISSANCE D’UN REEL:
%1) Définition :
%\textbf{\exemple}

%\textbf{\solution}

%\textbf{\remarque}
%\subsection*{\underline{\textbf{\textcolor{red}{2) Propriétés :}}}}
%\textbf{\exerciceapp}

%\textbf{\correction}
%III) CALCULS SUR LES RADICAUX:
%1) Définition :
%\textbf{\exemple}

%\textbf{\solution}
%\subsection*{\underline{\textbf{\textcolor{red}{2) Propriétés :}}}}
%\textbf{\remarque}
\end{document}
