\documentclass[12pt]{article}
\usepackage{stmaryrd}
\usepackage{graphicx} % Pour l'insertion d'images
\usepackage{float}    % Pour contrôler précisément le placement
\usepackage[utf8]{inputenc}

\usepackage[french]{babel}
\usepackage[T1]{fontenc}
\usepackage{hyperref}
\usepackage{verbatim}

\usepackage{color, soul}

\usepackage{pgfplots}
\pgfplotsset{compat=1.15}
\usepackage{mathrsfs}

\usepackage{amsmath}
\usepackage{amsfonts}
\usepackage{amssymb}
\usepackage{tkz-tab}
\author{Destiné aux élèves de $2^{nd}$ L\\Lycée de Dindéfelo\\Présenté par M. BA}
\title{\textbf{Calcul dans $\mathbb{R}$}}
\date{\today}
\usepackage{tikz}
\usetikzlibrary{arrows, shapes.geometric, fit}

% Commande pour la couleur d'accentuation
\newcommand{\myul}[2][black]{\setulcolor{#1}\ul{#2}\setulcolor{black}}
\newcommand\tab[1][1cm]{\hspace*{#1}}

\usepackage[margin=2cm]{geometry}
\usepackage{eso-pic}         % Pour ajouter des éléments en arrière-plan

\usepackage{enumitem}
%---------------------------------------
% Définir un compteur pour les exemples
\newcounter{exemple}

% Définir la commande \exemple pour afficher un exemple numéroté
\newcommand{\exemple}{%
  \refstepcounter{exemple}% Incrémenter le compteur
  \textbf{\textcolor{orange}{Exemple \theexemple : }} \ignorespaces
}
%---------------------------------------
\newcounter{solution}

% Définir la commande \solutione pour afficher un solution numéroté
\newcommand{\solution}{%
  \refstepcounter{solution}% Incrémenter le compteur
  \textbf{\textcolor{orange}{Solution \thesolution : }} \ignorespaces
}
%---------------------------------------
\definecolor{myorange}{rgb}{1.0, 0.8, 0.0}

% Définir un compteur pour les exercices d'application
\newcounter{exerciceapp}

% Définir la commande pour afficher un exercice d'application numéroté
\newcommand{\exerciceapp}{%
  \refstepcounter{exerciceapp}%
  \textbf{\textcolor{myorange}{Exercice d'application \theexerciceapp :}} \ignorespaces
}
%--------------------------------------
% Définir un compteur pour les exercices d'application
\newcounter{correction}

% Définir la commande pour afficher un correction exercice d'application numéroté
\newcommand{\correction}{%
  \refstepcounter{correction}%
  \textbf{\textcolor{myorange}{Correction \thecorrection :}} \ignorespaces
}
%--------------------------------------
% Définir un compteur pour les remarque d'application
\newcounter{remarque}

%----------------------------------------
\definecolor{myorange1}{rgb}{1.0, 1.5, 0}
% Définir la commande pour afficher un remarque numéroté
\newcommand{\remarque}{%
  \refstepcounter{remarque}%
  \textbf{\textcolor{myorange1}{Remarque \theremarque :}} \ignorespaces
}
% Commande pour ajouter du texte en arrière-plan
\AddToShipoutPicture{
    \AtTextCenter{%
        \makebox[0pt]{\rotatebox{45}{\textcolor[gray]{0.9}{\fontsize{5cm}{5cm}\selectfont Pathé Gobel BA}}}
    }
}

\begin{document}
\maketitle
\section*{\underline{\textbf{\textcolor{red}{I. CALCULS SUR LES QUOTIENTS :}}}}
\subsection*{\underline{\textbf{\textcolor{red}{1.Règles de calcul :}}}}
\subsubsection*{\underline{\textbf{\textcolor{red}{a.Quotient de deux réels :}}}}
\textbf{\remarque}

\subsubsection*{\underline{\textbf{\textcolor{red}{b) Propriétés :}}}}

Soit $a$, $c$, $e$ des réels quelconques et $b$, $d$, $f$ des réels non nuls ($b \neq 0$, $d \neq 0$, $f \neq 0$) :

$\ast \ a = \frac{a}{b} \cdot b = 1 ; \frac{0}{b} = 0 ; \frac{a}{a} = 1 ; \frac{1}{\frac{a}{b}} = \frac{b}{a} \quad (\text{pour } a \neq 0).$

$-\frac{a}{b} = -\frac{a}{b} = \frac{-a}{b} = -\frac{a}{-b} = \frac{a}{-b}.$

\textbf{Addition-Soustraction}

$\frac{a}{b} + \frac{c}{b} =$

$\frac{a}{b} + c =$

$ \frac{a}{b} - \frac{c}{d} = \frac{a \times d - c \times b}{b \times d} $

$-\frac{a}{b} + c =$

$ \frac{a}{b} + \frac{c}{d} + \frac{e}{f} = \frac{a \times d \times f + c \times b \times f + e \times b \times d}{b \times d \times f}.$

\textbf{Produit}

$\frac{a}{b} \times \frac{c}{d}=$

$a \times \frac{c}{d}=$

\textbf{Quotient}

$\frac{\frac{a}{b}}{\frac{c}{d}}=$

$\frac{1}{\frac{c}{d}}=$

Egalité de deux quotients :

$\frac{a}{b}=\frac{c}{d} \implies$

\textbf{\exerciceapp}

Soit $a = 6$, $b = 3$, $c = 8$, $d = 4$, $e = 10$, et $f = 5$.

Utilise les propriétés des fractions pour effectuer les calculs suivants :

\begin{enumerate}
    \item \textbf{Addition-Soustraction :}
        \begin{enumerate}
            \item Calcule $\frac{a}{b} + \frac{c}{b}$.
            \item Calcule $\frac{a}{b} - \frac{c}{d}$.
            \item Calcule $\frac{a}{b} + \frac{c}{d} + \frac{e}{f}$.
        \end{enumerate}
    
    \item \textbf{Produit :}
        \begin{enumerate}
            \item Calcule $\frac{a}{b} \times \frac{c}{d}$.
            \item Calcule $a \times \frac{c}{d}$.
        \end{enumerate}
    
    \item \textbf{Quotient :}
        \begin{enumerate}
            \item Calcule $\frac{\frac{a}{b}}{\frac{c}{d}}$.
            \item Calcule $\frac{1}{\frac{c}{d}}$.
        \end{enumerate}
\end{enumerate}

\textbf{\correction}

\begin{enumerate}
    \item \textbf{Addition-Soustraction :}
        \begin{enumerate}
            \item $\frac{6}{3} + \frac{8}{3} = \frac{6 + 8}{3} = \frac{14}{3}$.
            \item $\frac{6}{3} - \frac{8}{4} = \frac{6 \times 4 - 8 \times 3}{3 \times 4} = \frac{24 - 24}{12} = 0$.
            \item $\frac{6}{3} + \frac{8}{4} + \frac{10}{5} = 2 + 2 + 2 = 6$.
        \end{enumerate}
    
    \item \textbf{Produit :}
        \begin{enumerate}
            \item $\frac{6}{3} \times \frac{8}{4} = \frac{6 \times 8}{3 \times 4} = \frac{48}{12} = 4$.
            \item $6 \times \frac{8}{4} = 6 \times 2 = 12$.
        \end{enumerate}
    
    \item \textbf{Quotient :}
        \begin{enumerate}
            \item $\frac{\frac{6}{3}}{\frac{8}{4}} = \frac{2}{2} = 1$.
            \item $\frac{1}{\frac{8}{4}} = \frac{1}{2}$.
        \end{enumerate}
    

\end{enumerate}

\textbf{**Pour (b) et (c) il est préférable de simplifier avant de fair des calculs**}

    \textbf{Égalité de deux quotients :}
    
    Vérifie si les quotients suivants sont égaux : $\frac{a}{b}$ et $\frac{c}{d}$. Justifie ta réponse.
    
 	   \textbf{Égalité de deux quotients :}
    
    $\frac{6}{3} = 2$ et $\frac{8}{4} = 2$, donc $\frac{6}{3} = \frac{8}{4}$.
    
\subsection*{\underline{\textbf{\textcolor{red}{2. Identités remarquables :}}}}
\subsubsection*{\underline{\textbf{\textcolor{red}{a. Rappels :}}}}
\begin{itemize}
    \item \textbf{Carré d'une somme :} \quad $(a + b)^2 = a^2 + 2ab + b^2$
    \item \textbf{Carré d'une différence :} \quad $(a - b)^2 = a^2 - 2ab + b^2$
    \item \textbf{Produit d'une somme par une différence :} \quad $(a + b)(a - b) = a^2 - b^2$
\end{itemize}

\subsubsection*{\underline{\textbf{\textcolor{red}{b) Autres identités remarquables :}}}}
\begin{itemize}
    \item \textbf{Cube d'une somme :} \quad $(a + b)^3 = a^3 + 3a^2b + 3ab^2 + b^3$
    \item \textbf{Cube d'une différence :} \quad $(a - b)^3 = a^3 - 3a^2b + 3ab^2 - b^3$
    \item \textbf{Somme des cubes :} \quad $a^3 + b^3 = (a + b)(a^2 - ab + b^2)$
    \item \textbf{Différence des cubes :} \quad $a^3 - b^3 = (a - b)(a^2 + ab + b^2)$
\end{itemize}

\textbf{\exerciceapp}

Simplifie et développe les expressions suivantes en utilisant les identités remarquables :

\begin{enumerate}
    \item \textbf{Carré d'une somme :}
        \begin{enumerate}
            \item $(x + 5)^2$
            \item $(2y + 3)^2$
        \end{enumerate}
    
    \item \textbf{Carré d'une différence :}
        \begin{enumerate}
            \item $(x - 7)^2$
            \item $(3a - 4)^2$
        \end{enumerate}
    
    \item \textbf{Produit d'une somme par une différence :}
        \begin{enumerate}
            \item $(x + 6)(x - 6)$
            \item $(5b + 2)(5b - 2)$
        \end{enumerate}
    
    \item \textbf{Cube d'une somme et d'une différence :}
        \begin{enumerate}
            \item $(x + 2)^3$
            \item $(y - 3)^3$
        \end{enumerate}
    
    \item \textbf{Somme et différence des cubes :}
        \begin{enumerate}
            \item $x^3 + 8$
            \item $27 - y^3$
        \end{enumerate}
\end{enumerate}

\textbf{\correction}

\begin{enumerate}
    \item \textbf{Carré d'une somme :}
        \begin{enumerate}
            \item $(x + 5)^2 = x^2 + 2 \times 5x + 5^2 = x^2 + 10x + 25$
            \item $(2y + 3)^2 = (2y)^2 + 2 \times 2y \times 3 + 3^2 = 4y^2 + 12y + 9$
        \end{enumerate}
    
    \item \textbf{Carré d'une différence :}
        \begin{enumerate}
            \item $(x - 7)^2 = x^2 - 2 \times 7x + 7^2 = x^2 - 14x + 49$
            \item $(3a - 4)^2 = (3a)^2 - 2 \times 3a \times 4 + 4^2 = 9a^2 - 24a + 16$
        \end{enumerate}
    
    \item \textbf{Produit d'une somme par une différence :}
        \begin{enumerate}
            \item $(x + 6)(x - 6) = x^2 - 6^2 = x^2 - 36$
            \item $(5b + 2)(5b - 2) = (5b)^2 - 2^2 = 25b^2 - 4$
        \end{enumerate}
    
    \item \textbf{Cube d'une somme et d'une différence :}
        \begin{enumerate}
            \item $(x + 2)^3 = x^3 + 3x^2 \times 2 + 3x \times 2^2 + 2^3 = x^3 + 6x^2 + 12x + 8$
            \item $(y - 3)^3 = y^3 - 3y^2 \times 3 + 3y \times 3^2 - 3^3 = y^3 - 9y^2 + 27y - 27$
        \end{enumerate}
    
    \item \textbf{Somme et différence des cubes :}
        \begin{enumerate}
            \item $x^3 + 8 = (x + 2)(x^2 - 2x + 4)$
            \item $27 - y^3 = (3 - y)(9 + 3y + y^2)$
        \end{enumerate}
\end{enumerate}

\section*{\underline{\textbf{\textcolor{red}{II. PUISSANCE D'UN RÉEL :}}}}
\subsection*{\underline{\textbf{\textcolor{red}{1. Définition :}}}}

Pour tout réel non nul $a$ ($a \neq 0$) et tout entier relatif $m$ ($m \in \mathbb{Z}$), on note $a^m$ et on lit « $a$ puissance $m$ » ou « $a$ exposant $m$ ».

Si $m \geq 2$, alors $a^m = a \times a \times a \times \dots \times a$, avec $m$ facteurs de $a$.


\textbf{\exemple}

\[
7^4 = 7 \times 7 \times 7 \times 7 \quad ; \quad (-2)^3 = (-2) \times (-2) \times (-2) \quad ; \quad (x - 3)^2 = (x - 3) \times (x - 3).
\]

\textbf{\remarque}

\begin{itemize}
    \item Si $m = 1$, alors $a^m = a^1 = a$.
    \item Si $m = 0$, alors $a^m = a^0 = 1$.
    \item Si $m < -1$, alors $a^m = \frac{1}{a^{-m}}$.
    \item Si $m \geq 1$, alors $0^m = 0$.
    \item Si $m = 0$, alors $0^0$ n'existe pas.
    \item Si $m < -1$, alors $0^m$ n'existe pas.\\
     \textbf{Exemples :} $0^{-3}$ ; $0^{-5}$ ; $0^{-10}$ n'existent pas.
\end{itemize}

\subsection*{\underline{\textbf{\textcolor{red}{2) Propriétés :}}}}

Quelques soient les réels non nuls $a$ et $b$ ($a \neq 0 ; b \neq 0$) et quelques soient les entiers relatifs non nuls $m$ et $n$ ($m \neq 0 ; n \neq 0$), on a :

\begin{itemize}
    \item $\ast \ a^m \times a^n = a^{m+n}$ \quad ; \quad $(a \times b)^n = a^n \times b^n$ \quad ; \quad $(a^m)^n = a^{m \times n}$.
    \item $\ast \ \frac{a^m}{a^n} = a^{m-n}$ \quad ; \quad $\left(\frac{a}{b}\right)^n = \frac{a^n}{b^n}$ \quad ; \quad $\frac{1}{a^n} = a^{-n}$.
\end{itemize}

%\textbf{\exemple}

%\textbf{\solution}

%\textbf{\remarque}
%\subsection*{\underline{\textbf{\textcolor{red}{2) Propriétés :}}}}
%\textbf{\exerciceapp}

%\textbf{\correction}

\section*{\underline{\textbf{\textcolor{red}{III) CALCULS SUR LES RADICAUX:}}}}
\subsection*{\underline{\textbf{\textcolor{red}{1. Définition :}}}}
Soit \( a \) un réel positif ou nul (\( a \geq 0 \)).On appelle racine carrée de \( a \), notée \( \sqrt{a} \) ; le réel positif ou nul dont le carré est égale à \( a \) : \( \left( \sqrt{a}\right) ^{2}=a \).

\textbf{\exemple}

\( \sqrt{5}^{2}=5 \) ; \( \sqrt{19}^{2}=19 \)

Attention !! On n’écrit pas la racine carrée d’un nombre négatif : 

Par exemples \( \sqrt{-2} \) ; \( \sqrt{-7} \)

\subsection*{\underline{\textbf{\textcolor{red}{2) Propriétés :}}}}

Soient \( a \) , \( b \) et \( d \) deux réels strictement positifs ( \( a \geq 0 ; b > 0 ; d > 0\) ) et soit \( n \) un entier naturel, on a :
\begin{itemize}
\item[$\ast$] \( \sqrt{a \times b} = \sqrt{a} \times \sqrt{b} \)
\item[$\ast$] \( (\sqrt{a})^{2} = a \) 
\item[$\ast$]	\( (\sqrt{a^{2}}) = |a| \) 
\item[$\ast$]	\( (\sqrt{a})^{n} = ... \) 
\item[$\ast$] \( \frac{1}{\sqrt{a}}=\frac{\sqrt{a}}{a} \)
\item[$\ast$] \( \sqrt{\frac{a}{b}}=\frac{\sqrt{a}}{\sqrt{b}} \)
\item[$\ast$] \( d\sqrt{a}+c\sqrt{a} = (d+c)\sqrt{a} \)
\item[$\ast$] \( (d\sqrt{a})^{2} = (d)^{2}(\sqrt{a})^{2} \)
\item[$\ast$] \(\text{l’expression conjuguée de } \sqrt{a}+\sqrt{b} \text{ est } \sqrt{a}-\sqrt{b} \)
\item[$\ast$] \(\text{l’expression conjuguée de } \sqrt{a}-\sqrt{b} \text{ est } \sqrt{a}+\sqrt{b} \)
\end{itemize}
\section*{\underline{\textbf{\textcolor{red}{IV) INTERVALLES DANS $\mathbb{R}$:}}}}
\subsection*{\underline{\textbf{\textcolor{red}{1) Définition :}}}}
L’ensemble des abscisses des points d’une droite graduée est appelé ensemble des
nombres réels, que l’on note
\subsection*{\underline{\textbf{\textcolor{red}{2)Présentation des différents types d’intervalles :}}}}

\begin{center}
\begin{tabular}{|c|c|c|}
\hline
\textbf{Type d'intervalle} & \textbf{Exemples} \\
\hline
Intervalles bornés & $\left[a, b\right]$, $\left]a, b\right[$, $\left[a, b\right[$, $\left]a, b\right]$ \\
\hline
Intervalles non bornés & $\left]-\infty, a\right]$, $\left]a, +\infty\right[$ \\
\hline
\end{tabular}
\end{center}
\textbf{\remarque}

Soit $a$ un réel.

\begin{itemize}
\item[$\star$] $\left[a, a\right]={a}$ on lit "singleton $a$" est un intervalle. Exemples: $\left[4, 4\right]={4}$ et $\left[-2, -2\right]={-2}$

\item[$\star$] $\left]a, a\right[=\emptyset$ Exemples: $\left]4, 4\right[=\emptyset$ et $\left]-2, -2\right[=\emptyset$
\end{itemize}

\subsection*{\underline{\textbf{\textcolor{red}{3)Intersection et réunion de deux intervalles :}}}}
\subsection*{\underline{\textbf{\textcolor{red}{a) Intersection de deux intervalles :}}}}
L’intersection de deux intervalles $I$ et $J$, est l’ensemble des réels appartenant à $I$ et à $J$.

On la note $I \cap J$ et on lit « $I$ inter $J$ ».

\subsection*{\underline{\textbf{\textcolor{red}{b) Réunion de deux intervalles :}}}}
La réunion de deux intervalles $I$ et $J$, est l’ensemble des réels appartenant à $I$ ou bien à $J$.
On la note $I \cup J$ et on lit « $I$ union $J$ ».
\section*{\underline{\textbf{\textcolor{red}{V) VALEUR ABSOLUE:}}}}
\subsection*{\underline{\textbf{\textcolor{red}{1) Définition :}}}}
On appelle valeur absolue du réel a, le réel positif noté |a|, définie ainsi :
\begin{itemize}
\item[$\star$] Si a est strictement positif (a > 0) alors |a| = a ;
\item[$\star$] Si a est strictement négatif (a < 0) alors |a| = -(a) ;
\item[$\star$] Si a est égale à zéro (a = 0) alors |a| = |0| = 0.
\end{itemize}
\end{document}