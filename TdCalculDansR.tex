\documentclass[12pt]{article}
\usepackage{lmodern} % Pour une police plus nette
\usepackage{stmaryrd}
\usepackage{graphicx} % Pour l'insertion d'images
\usepackage{float}    % Pour contrôler précisément le placement
\usepackage[utf8]{inputenc}
\usepackage[french]{babel}
\usepackage[T1]{fontenc}
\usepackage{hyperref}
\usepackage{verbatim}
\usepackage{color, soul}
\usepackage{pgfplots}
\pgfplotsset{compat=1.18} % Version plus récente de pgfplots
\usepackage{mathrsfs}
\usepackage{amsmath}
\usepackage{amsfonts}
\usepackage{amssymb}
\usepackage{tkz-tab}
\usepackage{enumitem}
%\author{Destiné aux élèves de Terminale S\\Lycée de Dindéfelo\\Présenté par M. BA}
%\title{\textbf{Rappels et compléments sur les fonctions numériques}}
%\date{\today}
\usepackage{tikz}
\usetikzlibrary{arrows, shapes.geometric, fit}
% Commande pour la couleur d'accentuation
\newcommand{\myul}[2][black]{\setulcolor{#1}\ul{#2}\setulcolor{black}}
\newcommand\tab[1][1cm]{\hspace*{#1}}
\usepackage[margin=2.5cm]{geometry} % Ajustement des marges
\usepackage{eso-pic} % Pour ajouter des éléments en arrière-plan

% Commande pour ajouter du texte en arrière-plan, centré au milieu de chaque page
\AddToShipoutPicture{
    \AtPageCenter{%
        \makebox(0,0)[c]{\rotatebox{60}{\textcolor[gray]{0.9}{\fontsize{2cm}{2cm}\selectfont PGB}}}
    }
}

\newcounter{exercice}

% Définir la commande \exemple pour afficher un exemple numéroté
\newcommand{\exercice}{%
  \refstepcounter{exercice}% Incrémenter le compteur
  \textbf{\textcolor{black}{Exercice \theexercice  }} \ignorespaces
}

\begin{document}

\noindent
\begin{minipage}[t]{0.48\textwidth}
\raggedright
\textbf{Ministère de l'Éducation Nationale}\\
Inspection Académique de Kédougou\\
Lycée Dindéfelo\\
Cellule de Mathématiques
\end{minipage}
\hfill
\begin{minipage}[t]{0.48\textwidth}
\raggedleft
\textbf{Année scolaire 2024-2025}\\
Date : 05/11/2024\\
Classe : 2nd L\\
Professeur : M. BA
\end{minipage}

\vspace{0.5cm}

\begin{center}
\textbf{Calcul dans $\mathbb{R}$}
\end{center}

\textbf{\underline{\exercice}:} Ecrire les réels suivants sous formes de fractions irréductibles :

\[
\textbf{A}=\frac{\frac{3}{2}-\frac{5}{3}}{-\frac{7}{2}+\frac{1}{6}} \times \frac{1}{-3+\frac{2}{5}} \quad\quad;\quad\quad \textbf{B}=\frac{2}{3}-\frac{4}{5}+\frac{3}{7} \quad\quad;\quad\quad \textbf{C}=\left(3+\frac{4}{5} \right) \left(\frac{1}{5}-4 \right) 
\]

\[
\textbf{D}=\left(\frac{2}{3}-1 \right)\times \frac{2+\frac{1}{3}}{\frac{3}{5}+\frac{1}{5}} \quad\quad;\quad\quad \textbf{E}=\frac{3+\frac{1}{3}}{\frac{2}{3}-2} \times \frac{2}{\frac{1}{3}+3} \quad\quad;\quad\quad \textbf{F}=\frac{\frac{1}{4}+3}{\frac{3}{5}} \times \frac{7}{3}
\]

\[
\textbf{G}= \frac{5}{2}-\frac{1}{\frac{2}{3}} \quad\quad;\quad\quad \textbf{H}= \frac{\frac{2}{3}-\frac{2}{3}}{\frac{5}{4}} - \frac{7}{15} \quad\quad;\quad\quad \textbf{I}= \frac{\frac{1}{2}+\frac{2}{3}-1}{\frac{3}{4}}
\]

\[
\textbf{J}= \left(5-\frac{3}{2} \right)  \left(\frac{3}{4}+\frac{-5}{4} \right)   \quad\quad;\quad\quad \textbf{K}=\left(\frac{5}{3}+1 \right) \div \left(\frac{4}{3}-\frac{1}{4} \right)   \quad\quad;\quad\quad \textbf{L}=\left(3-\frac{2}{3} \right) + \left(-\frac{3}{2}+3 \right)   
\]

\textbf{\underline{\exercice}:}

1) Développer et réduire les expressions données :

\[
A = (-5x - 4)(2x^2 + x - 3) + (2x - 3)^3
\]
\[
B = (1 + x)^3 - (2\sqrt{5} - 3x)(2\sqrt{5} + 3x)
\]
\[
C = (2x - 1)(x - 2)^3
\]
\[
D = (2x - 3)(4x^2 + 6x + 9)
\]
\[
E = (-2x + 3)^3 - 2x(2x^2 - 5x + 3)
\]
\[
F = (x - 2)^2(-2x^2 + x + 1)
\]

2) Factoriser au mieux :

\[
F = 4x^2 + 20x + 25 - 9(4x^2 - 25)
\]
\[
G = 27 - 8x^3
\]
\[
H = 4(x - 3)^2 - (3x + 1)^2
\]
\[
I = x^3 - 8 + 2(x - 2)^2
\]

\textbf{\underline{\exercice}:} Exprimer sous forme de fractions irréductibles :

\[
A = \frac{(-3)^{2} \times 5^3 \times {10}^2}{2^3 \times 6 \times 3^6 \times (5 \times 2^{-3})^2} \quad ; \quad
B = \frac{(-6)^2 \times (7^{-2})^3 \times (9)^4}{(3^5) \times (-5)^{4} \times (-5)^{2}} \quad ; \quad 
C = \frac{(-12)^3 \times (x^{-6})^{-3}}{(-9)^3 \times (x^{-3})^5}
\]

\[
D = \left( \frac{-27}{15}\right) ^{2} \times \left( \frac{(-3)^{2}}{5} \right) 
\quad ; \quad
E = \left( -3 \right)^4 \times \left( -5 \right) \times \left( -3 \right)^4 \times \frac{(-3)^{4}}{(-3)^{6}}\times\left[(-3)^{-2} \right]^{-1}
\]

\[
F = \frac{(-2)^5 \times  (-5^{8}) \times (-9^{3}) }{(-6)^4 \times (30)^5 } \quad ; \quad
G = (-2)^2 \times (-5) \times  \frac{(-3)^{3}}{5} \quad ; \quad H = \left( \frac{-2}{3} +2 \right) \times \frac{3}{2}
\]

\textbf{\underline{\exercice}:} simplifier l'écriture des réels suivants :

\[
\begin{aligned}
    A &= 7\sqrt{20} - 11\sqrt{45} + 3\sqrt{80} ; \\
    B &= 2\sqrt{18} - 5\sqrt{8} + 4\sqrt{50} ; \\
    C &= 2\sqrt{27} - \sqrt{75} + 8\sqrt{12} ; \\
    D &= 2\sqrt{20} - 5\sqrt{45} + 6\sqrt{5} - 2\sqrt{180} ; \\
    E &= \left( \sqrt{3} - 5 \right)^2 + (\sqrt{2} + 1)(\sqrt{2} - 1) + \sqrt{3}(5\sqrt{3} - 1) ; \\
    F &= (\sqrt{2} - 4)^2 + 5\sqrt{2}(3 + \sqrt{2}) - (7\sqrt{2} + 1)(7\sqrt{2} - 1) ; \\
    G &= \frac{-3\sqrt{5}}{2\sqrt{15}} ; \\
    H &= \frac{2}{\sqrt{3} - 1} ; \\
    I &= \frac{\sqrt{3} - \sqrt{5}}{\sqrt{7} - 1} \times \frac{\sqrt{3} + \sqrt{5}}{\sqrt{7} + 1} ; \\
    J &= \frac{(\sqrt{3} - 2)^{2}}{\sqrt{7}} \times \frac{7 + 4 \sqrt{3}}{\sqrt{14}} ; \\
    K &= \frac{2}{1 - \sqrt{3}} + \frac{1}{1 + \sqrt{3}} ; \\
    L &= \frac{\sqrt{16}}{28} - \frac{\sqrt{125}}{49} - \frac{\sqrt{25}}{7} .
\end{aligned}
\]

\textbf{\underline{\exercice}:}

1) Traduire les inégalités suivantes sous formes d'intervalles :
\[
\begin{aligned}
    &\text{a) } 2 < x \leq 4 ; \quad \text{b) } -3 < x < 21 ; \quad \text{c) } x < -5 ; \quad \text{d) } x \geq -3 ; \quad \text{e) } -4 < x < 25 ; \quad \text{f) } x < 0 ; \\
    &\text{g) } 0 \leq x \leq 5 ; \quad \text{h) } -1 < x \leq 15 ; \quad \text{i) } x < 3 ; \quad \text{j) } 3x \geq 9 ; \quad \text{k) } -2x < 8 .
\end{aligned}
\]

2) Traduire les intervalles suivants sous formes d'inégalités :
\[
\begin{aligned}
    &\text{a) } x \in ]-2; 5[ ; \quad \text{b) } x \in ]-\infty; 5] ; \quad \text{c) } x \in [-3; 10[ ; \quad \text{d) } x \in [-2; +\infty[ ; \quad \text{e) } x \in ]11; 20[ ; \\
    &\text{f) } x \in [0; +\infty[ ; \quad \text{g) } x \in ]-\infty; 5] .
\end{aligned}
\]

3) Déterminer \( A \cap B \) et \( A \cup B \) dans chacun des cas suivants :
\[
\begin{aligned}
    &\text{a) } A = [3; 10] \text{ et } B = [5; 7] ; \quad \text{b) } A = ]-\infty; 0] \text{ et } B = [0; +\infty[ ; \\
    &\text{c) } A = [2; +\infty[ \text{ et } B = ]-5; 2] ; \quad \text{d) } A = [2; 11] \text{ et } B = ]-\infty; 0] ; \\
    &\text{e) } A = [4; 7] \text{ et } B = ]-\infty; 5] ; \quad \text{f) } A = ]-\infty; -3] \text{ et } B = [-3; +\infty[ .
\end{aligned}
\]

\textbf{\underline{\exercice}:}

\[\text{a)} |5 - x| - |-x + 7| = 0 \quad;\quad \text{b)} |2x + 3| + 7 = 0 \quad;\quad \text{c)} |8x - 1| = 5 \quad;\quad \text{d)} |-x + 3| \leq 4 \]

\[\text{e)} |-x + 6| + 8 \geq 0 \quad;\quad \text{f)} |2x + 5| > 7 \quad;\quad \text{g)} |-3x - 2| < -9 \quad;\quad \text{h)} |-2x + 1| - 5 \leq 0 \]

\[\text{i)} |3x - 5| < 4 \quad;\quad \text{j)} |-3x + 2| = |4x + 6| \quad;\quad \text{k)} |-2x + 3| - |3x - 5| = 0 \]

\end{document}