\documentclass[12pt]{article}
\usepackage{lmodern} % Pour une police plus nette
\usepackage{stmaryrd}
\usepackage{graphicx} % Pour l'insertion d'images
\usepackage{float}    % Pour contrôler précisément le placement
\usepackage[utf8]{inputenc}
\usepackage[french]{babel}
\usepackage[T1]{fontenc}
\usepackage{hyperref}
\usepackage{verbatim}
\usepackage{color, soul}
\usepackage{pgfplots}
\pgfplotsset{compat=1.18} % Version plus récente de pgfplots
\usepackage{mathrsfs}
\usepackage{amsmath}
\usepackage{amsfonts}
\usepackage{amssymb}
\usepackage{tkz-tab}
\usepackage{enumitem}
%\author{Destiné aux élèves de Terminale S\\Lycée de Dindéfelo\\Présenté par M. BA}
%\title{\textbf{Rappels et compléments sur les fonctions numériques}}
%\date{\today}
\usepackage{tikz}
\usetikzlibrary{arrows, shapes.geometric, fit}
% Commande pour la couleur d'accentuation
\newcommand{\myul}[2][black]{\setulcolor{#1}\ul{#2}\setulcolor{black}}
\newcommand\tab[1][1cm]{\hspace*{#1}}
\usepackage[margin=2.5cm]{geometry} % Ajustement des marges
\usepackage{eso-pic} % Pour ajouter des éléments en arrière-plan

% Commande pour ajouter du texte en arrière-plan, centré au milieu de chaque page
\AddToShipoutPicture{
    \AtPageCenter{%
        \makebox(0,0)[c]{\rotatebox{60}{\textcolor[gray]{0.9}{\fontsize{2cm}{2cm}\selectfont PGB}}}
    }
}

\newcounter{exercice}

% Définir la commande \exemple pour afficher un exemple numéroté
\newcommand{\exercice}{%
  \refstepcounter{exercice}% Incrémenter le compteur
  \textbf{\textcolor{black}{Exercice \theexercice  }} \ignorespaces
}

\begin{document}

\noindent
\begin{minipage}[t]{0.48\textwidth}
\raggedright
\textbf{Ministère de l'Éducation Nationale}\\
Inspection Académique de Kédougou\\
Lycée Dindéfelo\\
Cellule de Mathématiques
\end{minipage}
\hfill
\begin{minipage}[t]{0.48\textwidth}
\raggedleft
\textbf{Année scolaire 2024-2025}\\
Date : 05/11/2024\\
Classe : 2nd L\\
Professeur : M. BA
\end{minipage}

\vspace{0.5cm}

\begin{center}
\textbf{Calcul dans $\mathbb{R}$}
\end{center}

\textbf{\underline{\exercice}:} Ecrire les réels suivants sous formes de fractions irréductibles :

\[
\textbf{A}=\frac{\frac{3}{2}-\frac{5}{3}}{-\frac{7}{2}+\frac{1}{6}} \times \frac{1}{-3+\frac{2}{5}} \quad\quad;\quad\quad \textbf{B}=\frac{2}{3}-\frac{4}{5}+\frac{3}{7} \quad\quad;\quad\quad \textbf{C}=\left(3+\frac{4}{5} \right) \left(\frac{1}{5}-4 \right) 
\]

\[
\textbf{D}=\left(\frac{2}{3}-1 \right)\times \frac{2+\frac{1}{3}}{\frac{3}{5}+\frac{1}{5}} \quad\quad;\quad\quad \textbf{E}=\frac{3+\frac{1}{3}}{\frac{2}{3}-2} \times \frac{2}{\frac{1}{3}+3} \quad\quad;\quad\quad \textbf{F}=\frac{\frac{1}{4}+3}{\frac{3}{5}} \times \frac{7}{3}
\]
\end{document}