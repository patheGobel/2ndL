\documentclass[12pt]{article}
\usepackage{lmodern} % Pour une police plus nette
\usepackage{stmaryrd}
\usepackage{graphicx} % Pour l'insertion d'images
\usepackage{float}    % Pour contrôler précisément le placement
\usepackage[utf8]{inputenc}
\usepackage[french]{babel}
\usepackage[T1]{fontenc}
\usepackage{hyperref}
\usepackage{verbatim}
\usepackage{color, soul}
\usepackage{pgfplots}
\pgfplotsset{compat=1.18} % Version plus récente de pgfplots
\usepackage{mathrsfs}
\usepackage{amsmath}
\usepackage{amsfonts}
\usepackage{amssymb}
\usepackage{tkz-tab}
\usepackage{enumitem}
%\author{Destiné aux élèves de Terminale S\\Lycée de Dindéfelo\\Présenté par M. BA}
%\title{\textbf{Rappels et compléments sur les fonctions numériques}}
%\date{\today}
\usepackage{tikz}
\usetikzlibrary{arrows, shapes.geometric, fit}
% Commande pour la couleur d'accentuation
\newcommand{\myul}[2][black]{\setulcolor{#1}\ul{#2}\setulcolor{black}}
\newcommand\tab[1][1cm]{\hspace*{#1}}
\usepackage[margin=2.5cm]{geometry} % Ajustement des marges
\usepackage{eso-pic} % Pour ajouter des éléments en arrière-plan

% Commande pour ajouter du texte en arrière-plan, centré au milieu de chaque page
\AddToShipoutPicture{
    \AtPageCenter{%
        \makebox(0,0)[c]{\rotatebox{60}{\textcolor[gray]{0.9}{\fontsize{2cm}{2cm}\selectfont PGB}}}
    }
}

\newcounter{exercice}

% Définir la commande \exemple pour afficher un exemple numéroté
\newcommand{\exercice}{%
  \refstepcounter{exercice}% Incrémenter le compteur
  \textbf{\textcolor{black}{Exercice \theexercice  }} \ignorespaces
}

\begin{document}

\noindent
\begin{minipage}[t]{0.48\textwidth}
\raggedright
\textbf{Ministère de l'Éducation Nationale}\\
Inspection Académique de Kédougou\\
Lycée Dindéfelo\\
Cellule de Mathématiques
\end{minipage}
\hfill
\begin{minipage}[t]{0.48\textwidth}
\raggedleft
\textbf{Année scolaire 2024-2025}\\
Date : 28/01/2025\\
Classe : 2nd L\\
Professeur : M. BA
\end{minipage}

\vspace{0.5cm}

\begin{center}
\textbf{Calcul dans $\mathbb{R}$}
\end{center}

\section*{Exercice 1 (Calcul du discriminant)}

Calculer le discriminant \(\Delta\) (délta) de chacun des trinômes suivants :

\[ A(x) = 2x^2 + 8x - 34 \quad\quad B(x) = -x^2 + 11x - 4  \quad\quad C(x) = 3x^2 - 6x - 7 \quad\quad D(x) = -\frac{3}{2}x^2 + 5x + \frac{9}{4} \] 

\[ E(x) = x^2 + 7x + 23 \quad\quad F(x) = -10x^2 + 2x - 3 \quad\quad G(x) = x^2 + x + 3  \quad\quad H(x) = -6x^2 - 7x - 75 \] 
   
\[I(x) = x^2 + 2x + 1 \quad\quad J(x) = x^2 + 10x + 25 \quad\quad  K(x) = x^2 - 30x + 225 \quad\quad L(x) = -x^2 - 6x - 9 \]      

\section*{Exercice 2 : (Forme canonique)}

Donner la forme canonique de chacun des trinômes de l’exercice 1.

\section*{Exercice 3 : (Factorisation d’un trinôme)}

Factoriser (si possible) chacun des trinômes suivants :

\[
M(x) = x^2 + 6x - 7 \quad\quad n(x) = -2x^2 + 4x + 6 \quad\quad O(x) = x^2 - 8x + 17 \quad\quad T(x) = -9x^2 + 6x - 1
\]

\[
P(x) = 3x^2 - x - 4 \quad\quad Q(x) = x^2 - 9x - 90 \quad\quad R(x) = 12x^2 - 13x - 25 \quad\quad S(x) = x^2 - 20x + 1000
\]

\[
T(x) = x^2 - 30x + 225 \quad\quad U(x) = x^2 + 2x + 7
\]

\section*{Exercice 4 : (Utilisation de \(\Delta\))}

Résoudre dans \(\mathbb{R}\) les équations suivantes :

\[
a) \, 2x^2 - 5x + 2 = 0 \quad\quad b) \, 3x^2 - 14x + 8 = 0 \quad\quad c) \, 5x^2 - 7x + 4 = 0 \quad\quad d) \, -4x^2 + 10x - 6 = 0
\]

\[
e) \, x^2 + 4x - 45 = 0 \quad\quad f) \, x^2 + 6x - 16 = 0 \quad\quad g) \, 4x^2 - x + 3 = 0 \quad\quad h) \, x^2 + 10x + 25 = 0
\]

\[
m) \, x^2 - 30x + 225 = 0 \quad\quad n) \, x^2 - x - 12 = 0 \quad\quad o) \, 6x^2 - 7x + 1 = 0 \quad\quad p) \, -5x^2 - 3x - 8 = 0
\]


\end{document}
