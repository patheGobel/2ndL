\documentclass[12pt]{article}
\usepackage{lmodern} % Pour une police plus nette
\usepackage{stmaryrd}
\usepackage{graphicx} % Pour l'insertion d'images
\usepackage{float}    % Pour contrôler précisément le placement
\usepackage[utf8]{inputenc}
\usepackage[french]{babel}
\usepackage[T1]{fontenc}
\usepackage{hyperref}
\usepackage{verbatim}
\usepackage{color, soul}
\usepackage{pgfplots}
\pgfplotsset{compat=1.18} % Version plus récente de pgfplots
\usepackage{mathrsfs}
\usepackage{amsmath}
\usepackage{amsfonts}
\usepackage{amssymb}
\usepackage{tkz-tab}
\usepackage{enumitem}
%\author{Destiné aux élèves de Terminale S\\Lycée de Dindéfelo\\Présenté par M. BA}
%\title{\textbf{Rappels et compléments sur les fonctions numériques}}
%\date{\today}
\usepackage{tikz}
\usetikzlibrary{arrows, shapes.geometric, fit}
% Commande pour la couleur d'accentuation
\newcommand{\myul}[2][black]{\setulcolor{#1}\ul{#2}\setulcolor{black}}
\newcommand\tab[1][1cm]{\hspace*{#1}}
\usepackage[margin=2.5cm]{geometry} % Ajustement des marges
\usepackage{eso-pic} % Pour ajouter des éléments en arrière-plan

% Commande pour ajouter du texte en arrière-plan, centré au milieu de chaque page
\AddToShipoutPicture{
    \AtPageCenter{%
        \makebox(0,0)[c]{\rotatebox{60}{\textcolor[gray]{0.6}{\fontsize{2cm}{2cm}\selectfont PGB}}}
    }
}

\newcounter{exercice}

% Définir la commande \exemple pour afficher un exemple numéroté
\newcommand{\exercice}{%
  \refstepcounter{exercice}% Incrémenter le compteur
  \textbf{\textcolor{black}{Exercice \theexercice  }} \ignorespaces
}

\begin{document}

\noindent
\begin{minipage}[t]{0.48\textwidth}
\raggedright
\textbf{Ministère de l'Éducation Nationale}\\
Inspection Académique de Kédougou\\
Lycée Dindéfelo\\
Cellule de Mathématiques
\end{minipage}
\hfill
\begin{minipage}[t]{0.48\textwidth}
\raggedleft
\textbf{Année scolaire 2024-2025}\\
Date : 28/01/2025\\
Classe : 2nd L\\
Professeur : M. BA
\end{minipage}

\vspace{0.5cm}

\begin{center}
\textbf{Equations Inéquations Systèmes}
\end{center}

\section*{Exercice 1 (Calcul du discriminant)}

Calculer le discriminant \(\Delta\) (délta) de chacun des trinômes suivants :

\[ A(x) = 2x^2 + 8x - 34 \quad\quad B(x) = -x^2 + 11x - 4  \quad\quad C(x) = 3x^2 - 6x - 7 \quad\quad D(x) = -\frac{3}{2}x^2 + 5x + \frac{9}{4} \] 

\[ E(x) = x^2 + 7x + 23 \quad\quad F(x) = -10x^2 + 2x - 3 \quad\quad G(x) = x^2 + x + 3  \quad\quad H(x) = -6x^2 - 7x - 75 \] 
   
\[I(x) = x^2 + 2x + 1 \quad\quad J(x) = x^2 + 10x + 25 \quad\quad  K(x) = x^2 - 30x + 225 \quad\quad L(x) = -x^2 - 6x - 9 \]      

\section*{Exercice 2 : (Forme canonique)}

Donner la forme canonique de chacun des trinômes de l’exercice 1.

\section*{Exercice 3 : (Factorisation d’un trinôme)}

Factoriser (si possible) chacun des trinômes suivants :

\[
M(x) = x^2 + 6x - 7 \quad\quad n(x) = -2x^2 + 4x + 6 \quad\quad O(x) = x^2 - 8x + 17 \quad\quad T(x) = -9x^2 + 6x - 1
\]

\[
P(x) = 3x^2 - x - 4 \quad\quad Q(x) = x^2 - 9x - 90 \quad\quad R(x) = 12x^2 - 13x - 25 \quad\quad S(x) = x^2 - 20x + 1000
\]

\[
T(x) = x^2 - 30x + 225 \quad\quad U(x) = x^2 + 2x + 7
\]

\section*{Exercice 3 : (Factorisation d’un trinôme)}

Nous allons calculer le discriminant \(\Delta\), déterminer les racines et factoriser chaque trinôme lorsque c'est possible.

\subsection*{1. \( M(x) = x^2 + 6x - 7 \)}
\[
\Delta = 6^2 - 4(1)(-7) = 36 + 28 = 64
\]
\[
x_{1,2} = \frac{-6 \pm \sqrt{64}}{2(1)} = \frac{-6 \pm 8}{2}
\]
\[
x_1 = 1, \quad x_2 = -7
\]
\[
M(x) = (x - 1)(x + 7)
\]

\subsection*{2. \( N(x) = -2x^2 + 4x + 6 \)}
\[
\Delta = 4^2 - 4(-2)(6) = 16 + 48 = 64
\]
\[
x_{1,2} = \frac{-4 \pm \sqrt{64}}{2(-2)} = \frac{-4 \pm 8}{-4}
\]
\[
x_1 = -1, \quad x_2 = 3
\]
\[
N(x) = -2(x + 1)(x - 3)
\]

\subsection*{3. \( O(x) = x^2 - 8x + 17 \)}
\[
\Delta = (-8)^2 - 4(1)(17) = 64 - 68 = -4
\]
Pas de racines réelles, donc pas de factorisation possible dans \(\mathbb{R}\).

\subsection*{4. \( T(x) = -9x^2 + 6x - 1 \)}
\[
\Delta = 6^2 - 4(-9)(-1) = 36 - 36 = 0
\]
\[
x = \frac{-6}{2(-9)} = \frac{1}{3}
\]
\[
T(x) = -9(x - \frac{1}{3})^2
\]

\subsection*{5. \( P(x) = 3x^2 - x - 4 \)}
\[
\Delta = (-1)^2 - 4(3)(-4) = 1 + 48 = 49
\]
\[
x_{1,2} = \frac{1 \pm \sqrt{49}}{2(3)} = \frac{1 \pm 7}{6}
\]
\[
x_1 = \frac{-6}{6} = -1, \quad x_2 = \frac{8}{6} = \frac{4}{3}
\]
\[
P(x) = (3x - 4)(x + 1)
\]

\subsection*{6. \( Q(x) = x^2 - 9x - 90 \)}
\[
\Delta = (-9)^2 - 4(1)(-90) = 81 + 360 = 441
\]
\[
x_{1,2} = \frac{9 \pm \sqrt{441}}{2(1)} = \frac{9 \pm 21}{2}
\]
\[
x_1 = 15, \quad x_2 = -6
\]
\[
Q(x) = (x - 15)(x + 6)
\]

\subsection*{7. \( R(x) = 12x^2 - 13x - 25 \)}
\[
\Delta = (-13)^2 - 4(12)(-25) = 169 + 1200 = 1369
\]
\[
x_{1,2} = \frac{13 \pm \sqrt{1369}}{2(12)} = \frac{13 \pm 37}{24}
\]
\[
x_1 = \frac{50}{24} = \frac{25}{12}, \quad x_2 = \frac{-24}{24} = -1
\]
\[
R(x) = (12x - 25)(x + 1)
\]

\subsection*{8. \( S(x) = x^2 - 20x + 1000 \)}
\[
\Delta = (-20)^2 - 4(1)(1000) = 400 - 4000 = -3600
\]
Pas de racines réelles, donc pas de factorisation possible dans \(\mathbb{R}\).

\subsection*{9. \( T(x) = x^2 - 30x + 225 \)}
\[
\Delta = (-30)^2 - 4(1)(225) = 900 - 900 = 0
\]
\[
x = \frac{30}{2} = 15
\]
\[
T(x) = (x - 15)^2
\]

\subsection*{10. \( U(x) = x^2 + 2x + 7 \)}
\[
\Delta = 2^2 - 4(1)(7) = 4 - 28 = -24
\]
Pas de racines réelles, donc pas de factorisation possible dans \(\mathbb{R}\).

\section*{Exercice 3 : (Factorisation d’un trinôme)}

\begin{align*}
M(x) &= x^2 + 6x - 7 = (x - 1)(x + 7) \\[5pt]
N(x) &= -2x^2 + 4x + 6 = -2(x + 1)(x - 3) \\[5pt]
O(x) &= x^2 - 8x + 17 \quad \text{(Non factorisable dans $\mathbb{R}$)} \\[5pt]
T(x) &= -9x^2 + 6x - 1 = -9 \left( x - \frac{1}{3} \right)^2 \\[5pt]
P(x) &= 3x^2 - x - 4 = (3x - 4)(x + 1) \\[5pt]
Q(x) &= x^2 - 9x - 90 = (x - 15)(x + 6) \\[5pt]
R(x) &= 12x^2 - 13x - 25 = (12x - 25)(x + 1) \\[5pt]
S(x) &= x^2 - 20x + 1000 \quad \text{(Non factorisable dans $\mathbb{R}$)} \\[5pt]
T(x) &= x^2 - 30x + 225 = (x - 15)^2 \\[5pt]
U(x) &= x^2 + 2x + 7 \quad \text{(Non factorisable dans $\mathbb{R}$)}
\end{align*}

\section*{Exercice 4 : (Utilisation de \(\Delta\))}

Résoudre dans \(\mathbb{R}\) les équations suivantes :

\[
a) \, 2x^2 - 5x + 2 = 0 \quad\quad b) \, 3x^2 - 14x + 8 = 0 \quad\quad c) \, 5x^2 - 7x + 4 = 0 \quad\quad d) \, -4x^2 + 10x - 6 = 0
\]

\[
e) \, x^2 + 4x - 45 = 0 \quad\quad f) \, x^2 + 6x - 16 = 0 \quad\quad g) \, 4x^2 - x + 3 = 0 \quad\quad h) \, x^2 + 10x + 25 = 0
\]

\[
m) \, x^2 - 30x + 225 = 0 \quad\quad n) \, x^2 - x - 12 = 0 \quad\quad o) \, 6x^2 - 7x + 1 = 0 \quad\quad p) \, -5x^2 - 3x - 8 = 0
\]

\section*{Exercice 5: (Sans utilisation de \(\Delta\))}

Résoudre dans \(\mathbb{R}\) les équations suivantes :

\[
1) \, 7x^2 + 34x = 0 \quad\quad 2) \, -3x^2 + 4x = 0 \quad\quad 3) \, -5x^2 - 70x = 0
\]

\[
4) \, x^2 + 3x = 0 \quad\quad 5) \, -x^2 - 5x = 0 \quad\quad 6) \, -x^2 - x = 0
\]

\[
7) \, 2x^2 - 3 = 0 \quad\quad 8) \, -x^2 + 1 = 0 \quad\quad 9) \, x^2 - 9 = 0
\]

\[
10) \, x^2 + 1 = 0 \quad\quad 11) \, -x^2 - 7 = 0 \quad\quad 12) \, 3x^2 + 11 = 0
\]

\section*{Exercice 6 : (Somme et Produit)}

1- Trouver (s’ils existent) deux nombres réels dont la somme est \( S \) et le produit \( P \).

\[
a) \, S = 3 \, \text{et} \, P = -10 \quad\quad b) \, S = -3 \, \text{et} \, P = 9
\]

\[
c) \, S = 5 \, \text{et} \, P = 6 \quad\quad d) \, S = -6 \, \text{et} \, P = 9
\]

2- Trouver deux nombres dont leur somme est \( 8 \) et leur produit \( 15 \).

3- Un champ rectangulaire a pour périmètre \( 25 \, \text{m} \) et pour surface \( 150 \, \text{m}^2 \). Déterminer les dimensions du champ.

\section*{Exercice 7 : (Somme et Produit)}

On considère le trinôme suivant : \( P(x) = -3x^2 + 5x + 4 \)

1- Montrer que \( P \) admet deux racines distinctes \( x_1 \) et \( x_2 \).

2- Sans calculer \( x_1 \) et \( x_2 \) déterminer :

\[
S = x_1 + x_2 ; \, P = x_1 \times x_2 ; \, \frac{1}{x_1} + \frac{1}{x_2} ; \, x_1^2 + x_2^2
\]

\section*{Exercice 8 : (Racine - Utilisation somme ou produit)}

Dans chacun des cas suivants, vérifier que \( \alpha \) est racine de \( P \). Trouver l’autre racine en utilisant la somme ou le produit.

\[
a) \, P(x) = 7x^2 - 4x - 11 ; \quad \alpha = -1
\]

\[
b) \, P(x) = 2x^2 - 3x - 2 ; \quad \alpha = 2
\]

\[
c) \, P(x) = 5x^2 - x - 4 ; \quad \alpha = 1
\]

\[
d) \, P(x) = -2x^2 - 5x + 3 ; \quad \alpha = \frac{1}{2}
\]

\section*{Exercice 9 : (Signe d’un trinôme)}

Étudier, suivant les valeurs de \( x \), le signe du trinôme \( P(x) \).

\[
a) \, P(x) = 4x^2 - x + 1 \quad\quad b) \, P(x) = -2x^2 + 7x - 3
\]

\[
c) \, P(x) = -9x^2 + 6x - 1 \quad\quad d) \, P(x) = x^2 - 5x + 6
\]

\section*{Exercice 10 : (Inéquations du second degré)}

Résoudre dans \(\mathbb{R}\) les inéquations suivantes :

\[
a) \, -x^2 - x + 12 \geq 0 \quad\quad b) \, 2x^2 - x - 3 > 0
\]

\[
c) \, -3x^2 + 4x - 2 \leq 0 \quad\quad d) \, 8x^2 + 34x + 21 < 0
\]

\[
e) \, -9x^2 + 12x - 4 \geq 0 \quad\quad f) \, 4 - 9x^2 \leq 0
\]

\section*{Exercice 11 : (Encore des inéquations)}

Résoudre dans \( \mathbb{R} \) les inéquations suivantes :

\[
a) \, (-x^2 - x + 12)(x - 5) \geq 0 \quad\quad b) \, (2x^2 - x - 3)(x^2 + 3x - 4) > 0
\]

\[
c) \, (-3x^2 + 4x - 2)(x + 1) \leq 0 \quad\quad d) \, 2(8x^2 + 34x + 21)(x + 7) < 0
\]

\[
e) \, -3(x + 3)(-9x^2 + 12x - 4) \geq 0 \quad\quad f) \, 4(-x + 6)(x^2 - 10x + 25) < 0
\]

\section*{Exercice 12 : (Encore des inéquations)}

Résoudre dans \( \mathbb{R} \) les inéquations suivantes :

\[
1) \, \frac{x^2 + 5x - 6}{x + 2} \geq 0 \quad\quad 2) \, \frac{x^2 + 10x + 25}{x^2 + x + 1} \leq 0
\]

\[
3) \, \frac{-x^2 + 3x - 2}{-x + 3} < 0
\]

\section*{Exercice 13}

1) Dans chacun des cas suivants, vérifie si le couple donné est solution du système ou non :
\begin{itemize}
    \item[a)] 
    \[
    \begin{cases}
        2x - 3y = 1 \\
        x + 4y = 1
    \end{cases}
    \quad ; \quad (2; 1)
    \]
    \item[b)]
    \[
    \begin{cases}
        -3x - y + 2 = 0 \\
        x - y = 3
    \end{cases}
    \quad ; \quad (1; -1)
    \]
\end{itemize}

2) Dans chacun des cas suivants, vérifie si le couple donné est solution de l’inéquation ou non :
\begin{itemize}
    \item[a)] 
    \(-x + y > -2 \quad ; \quad (0; 0)\)
    \item[b)] 
    \(2x + y + 3 \leq 0 \quad ; \quad (-1; 1)\)
    \item[c)]
    \(x + 2y \leq 0 \quad ; \quad (1; -\frac{1}{2})\)
\end{itemize}

\subsection*{EXERCICE 14 :}
Résoudre chacun des systèmes suivants par la méthode indiquée :

\[
\text{a)} \quad 
\begin{cases}
x - y = \frac{3}{4} \\
5x + 2y = 3
\end{cases}
\quad ; \quad
\text{b)} \quad
\begin{cases}
2x - y = 5 \\
x - 2y = 3
\end{cases}
\quad ; \quad
\text{c)} \quad
\begin{cases}
2x - 3y = 4 \\
x + y = 5
\end{cases}
\]

\[
\text{d)} \quad
\begin{cases}
3x - y + 2 = 0 \\
3x - y - 1 = 0
\end{cases}
\quad ; \quad
\text{e)} \quad
\begin{cases}
2y + x = 5 \\
-y + 7 = 4
\end{cases}
\quad ; \quad
\text{f)} \quad
\begin{cases}
3x - y = 2 \\
2x - y = 1
\end{cases}
\]

\[
\text{g)} \quad
\begin{cases}
x + y = 3 \\
-y + 4 = x - 2
\end{cases}
\quad ; \quad
\text{h)} \quad
\begin{cases}
x - y = 5 \\
x + 2y = 4
\end{cases}
\quad ; \quad
\text{i)} \quad
\begin{cases}
x - 3y = 11 \\
2x - y = 8
\end{cases}
\]

\[
\text{j)} \quad
\begin{cases}
-4x + y = 2 \\
3x - y = -4
\end{cases}
\quad ; \quad
\text{k)} \quad
\begin{cases}
2x - y = 3 \\
3x - y = 5
\end{cases}
\quad ; \quad
\text{l)} \quad
\begin{cases}
3x - 2y = 5 \\
2x + y = 1
\end{cases}
\]

\[
\text{m)} \quad
\begin{cases}
\frac{x}{2} + \frac{y}{3} = 7 \\
\frac{x}{3} + \frac{y}{2} = 8
\end{cases}
\quad ; \quad
\text{n)} \quad
\begin{cases}
\frac{x}{6} + \frac{y}{5} = 5 \\
\frac{2}{3} x - \frac{y}{2}= 8
\end{cases}
\quad ; \quad
\text{o)} \quad
\begin{cases}
\frac{3}{2}x + \frac{5}{4}y = 11 \\
x + 2y = 11
\end{cases}
\]

\[
\text{p)} \quad
\begin{cases}
x\sqrt{2} - 5y\sqrt{3} = 17 \\
x\sqrt{6} + y = \sqrt{3}
\end{cases}
\text{q)} \quad
\begin{cases}
2x - y - 4 = 0 \\
x + 5y - 13 = 0 \\
-3x + 2y + 5 = 0
\end{cases}
\quad ; \quad
\text{r)} \quad
\begin{cases}
-x + y - 3 = 0 \\
-2x + 2y + 1 = 0 \\
3x - 2y - 4 = 0
\end{cases}
\]

1) Méthode de substitution : a) ; b) ; c) et d) 

2) Méthode d’addition : e) ; f) ; g) ; h) ; q) et r)

3) méthode graphique : i) ; j) ; k) et l)

4) méthode de CRAMER : m) ; n) ; o) ; p) ; q) et r).
\section*{Exercice 15}

Résous graphiquement, chacun des systèmes d’inéquations suivants :
\begin{itemize}
    \item[a)] 
    \[
    \begin{cases}
        2x + y \geq 1 \\
        -x + 2y \leq 3
    \end{cases}
    \]
    \item[b)]
    \[
    \begin{cases}
        x + y + 1 < 0 \\
        x - y - 2 \leq 0
    \end{cases}
    \]
    \item[c)]
    \[
    \begin{cases}
        x \geq -1 \\
        y < 0
    \end{cases}
    \]
    \item[d)]
    \[
    \begin{cases}
        -x - y + 1 < 0 \\
        y > \frac{1}{2}
    \end{cases}
    \]
    \item[e)]
    \[
    \begin{cases}
        2x - y \leq 2 \\
        x + 2y + 1 > 2
    \end{cases}
    \]
\end{itemize}
\end{document}
